\chapter{典型混沌系统的Koopman分析}

\section{Tent映射}
\begin{equation}
    x_{n+1}=f(x_n)=1-2|x-\frac{1}{2}| ,\ x_n\in [0,1], n=1,2,3,\cdots
\end{equation}

\section{Logistic映射}
Logistic映射来源于生态学中的虫口模型,其动力学方程可描述为
\begin{equation}
    x_{n+1}=f(x_n)=\gamma x_n(1-x_n),\ x_n\in [0,1], n=1,2,3,\cdots
\end{equation}

通过非线性动力学的知识可以得到,该系统在$\gamma$取不同值时表现出不同的动力学行为。当$0<\gamma<1$时,系统最终都会渐进的趋于0;当$1<\gamma<3$时,系统会收敛到一个不动点,该不动点的值为$x^*=(\gamma-1)/\gamma$,此时系统的极限行为会趋于该不动点的值;当$3<\gamma<3.57$时,系统的迭代会出现周期行为,随着r的增大,周期的长度也会相应的增加,例如2周期,4周期,8周期等;当$3.57<\gamma<4$时,系统的迭代会在周期类型和混沌类型之间来回切换;直到$\gamma=4$时,系统处于完全混沌的状态。在我们的Koopman分析中,我们取$\gamma=4$的一个特例,通过Logistic映射的动力学方程演化出一系列的数据,作为Koopman分析的源数据,以此来分析Logistic映射的系统特征。

\section{Henon映射}
Henon映射是由法国数学家Michel H\'{e}non提出的,以此作为Lorenz模型的Poincar\'{e}界面的简化模型。Henon映射是一个可以产生混沌现象的离散时间动态系统,其迭代方程可以描述为

\begin{equation}
    \begin{cases}
        x_{n+1}=y_n+1-ax_n^2\\
        y_{n+1}=bx_n
    \end{cases}\ x,y\in [-1.5,1.5]
\end{equation}

在经典的Henon映射中,我们通常取$a=1.4$与$b=0.3$,此时系统表现出混沌现象。在我们的Koopman分析中,我们在取上述参数值的情况下,通过Henon映射的迭代方程演化出一系列的数据,作为Koopman分析的源数据,以此来分析Henon映射的系统特征。


\section{Lorenz系统}
Lorenz系统是1963年由Edward Lorenz提出的描述空气流体运动的一个数学模型,通常用Lorenz方程来描述
\begin{equation}
    \begin{cases}
        \dot{x}=\sigma(y-x)\\
        \dot{y}=x(\rho-z)-y\\
        \dot{z}=xy-\beta z
    \end{cases}\
\end{equation}

该方程描述了一个三维相空间的三个分量对时间的变化率,是一个连续时间动态系统。Lorenz系统具有非线性、非周期性和确定性的性质。参数$\sigma$,$\rho$,$\beta$取不同值时Lorenz系统表现出不同的动力学行为。如在$\rho<1$时,系统只有一个不动点,即原点,所有轨道的长期行为都趋于原点;当$\rho=1$时系统发生了叉式分叉,在$\rho>1$时出现了两个不动点,不动点的稳定性可由其他参数满足的条件确定;当我们将三个参数取一组特定的值$\rho=28$,$\sigma=10$,$\beta=8/3$时,Lorenz方程的解是混沌的,相空间会存在两个奇异吸引子。Lorenz系统的这种动力学特征用传统的非线性动力学知识是比较难分析的,我们将在取上述参数的条件下,通过Lorenz微分方程演化出相空间的一条轨道,作为Koopman分析的源数据,以此来分析Lorenz系统的特征。

%% 本章参考文献
% \ifx\usechapbib\empty
% \nocite{BSTcontrol}
% \setcounter{NAT@ctr}{0}
% \bibliographystyle{buptgraduatethesis}
% \bibliography{references}
% \fi
