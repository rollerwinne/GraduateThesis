\chapter{引言}

当今时代是一个信息时代,社会的发展充斥着各种信息交换,包括机械、能源、交通、化工、生物、航空航天等各项领域,然而在信息传递中愈发庞大的信息量中,如何有效的提取关键的信息是一个亟待解决的问题。

信息的具体表现形式是数据,数据和信息直接是相互联系的,数据是反映客观事物属性的记录。数据经过加工处理之后,就成为信息,而信息需要经过数字化转变成数据才能存储和传输。因此,数据分析也成为当下的一个热点问题。

随着数学与计算机等领域的发展,测量技术和计算能力的巨大改进,数据分析已成为一门新兴的学科应用在各个领域,并将继续发展。然而数据分析的方法多种多样,不同的数据分析方法的有效性也不同,很多数据的分析方法都依赖于数据的数学特征。我们寄希望于找到一个普遍的数据分析方法,通过研究数据从而了解其内部的抽象结构。

在许多实际系统中,我们通常可以得到一系列关于该系统的丰富数据,然而由于不了解其动力学原理或者系统过于复杂等原因无法用比较准确的动力学方程来近似描述。为了了解和研究这些系统,通常进行大量的实验或观测得到系统的特性数据。我们希望能够找到一种方法,仅通过系统数据的演化过程得到系统的一些演化特征,并在这些演化特征中提取出关键特征。如果我们能提取出复杂系统的关键特征,就能通过分析实际复杂系统的长期行为,并对系统的行为作出一定程度的预测。在我们的现实生活中,若将各种事件抽象为一种数据,并能提取出关键特征并进行预测,将具有很大的现实意义。

\section{Koopman算符与动力系统的特征提取}

我们希望能够找到一种方法,仅通过系统数据的演化过程得到系统的一些演化特征,并在这些演化特征中提取出关键特征。Koopman算符给我们提供了一个有效的数学工具。Koopman算法由B.O.Koopman与1931年引入,它作用在某个函数上,描述了函数的演化。若将系统的特性数据演化视为函数的演化,我们即可以用Koopman算符分析系统的特征。

Koopman分析方法仅由系统的特性数据得到,理论上在数据足够多的情况下,我们可以得到系统的一切特征,如何提取出关键的系统特征(或者说是我们想要的系统特征)也是一个很复杂的问题;另一方面,我们在实际应用中的数据总是有限的,也会存在一定的误差,于是复杂系统的Koopman分析依赖于大量的系统数据支持。为了简化计算复杂度,而又不失Koopman分析的核心思想,我们可以先选取某些已知的动力学系统作为Koopman分析的对象。

线性的动力学系统通过经典的动力学理论已经能够得到较好的解析分析方法,而非线性动力学系统的分析方法也有诸多理论用于系统的分析,比如通过不动点的线性化分析得到不动点近邻区域的系统演化特征。而对于复杂的非线性动力学系统比如混沌系统,传统的非线性动力学分析方法有其局限性。我们试图将Koopman分析应用在某些混沌系统上,提取出混沌系统的系统特征,如果能抓住混沌系统的主要特征并进行定量的分析,就可以分析出混沌系统的长期行为,从而对混沌系统做一定程度上的预测。

混沌系统按照系统方程可以分为混沌映射(离散混沌系统)与混沌流(连续混沌系统),其本质是相同的。但由于在分析过程中需要用到数值计算,而数值计算对混沌映射与混沌流的影响是不同的:混沌映射的数值计算误差来源主要依赖于计算机的精度误差;混沌流的数值计算误差来源除了计算机的精度误差,其主要误差来源于系统时间间隔的选取。另一方面,混沌流的数值计算过程也可以转变为一个关于系统时间间隔的映射系统。因此,我们的Koopman分析方法以混沌映射系统为基础,分别讨论混沌映射系统与混沌流系统。

混沌系统按照系统维度可以分为一维、二维、三维与高维系统,不同维度的混沌系统在计算方法与计算复杂度上有一定区别。有一些常见的混沌系统,比如Logistic映射(一维映射),Henon映射(二维映射),Standard映射(二维映射),Lorenz系统(三维流),我们将选取几个典型的混沌系统作为分析对象,以验证Koopman分析的有效性。

通过对上述一些常见的混沌系统的Koopman分析,我们可以构造出每个系统下的数据矩阵,进一步计算出Koopman算符的矩阵表示以及本征值、本征函数。每个本征函数都代表了该系统的一种动力学模式,我们可以得到每个系统的很多本征函数,且本征函数的数量与相空间的数据点的数量相同,然而很多本征函数都是高度抽象的,很多本征函数反映了相同的动力学模式,而我们的工作就是提取出每个系统不同的动力学模式。

在混沌系统中的Koopman分析可以为我们在以后实际应用中更复杂的非线性系统打下基础。通过已知的混沌系统的Koopman分析方式,我们可以观察到计算Koopman算符的本征函数时,不同的参数对本征函数的影响,比如基函数的形式、基函数的数量、基函数的分布、演化格点的分布等。在了解不同的参数对混沌系统的影响后,我们可以将Koopman分析方法应用到实际的复杂系统中。在实际的复杂系统中,我们可以采用与混沌系统同样的分析方法,即使系统非常复杂,我们也无需知道系统的工作内部机制以及动力学方程,只需要得到系统的状态演化过程,便能对复杂系统进行Koopman分析,并提取复杂系统的关键特征,从而分析复杂系统的长期行为,并对某些系统行为做出预测。

\section{国内外的研究现状}
关于数据分析与数据预测的方法有诸多理论。在本研究课题中采取Koopman算符进行数据的分析与预测。Koopman 算符由B. O. Koopman 于1931 年提出,能够描述系统可观察量随时间的演化,并可用来研究动力系统的各态历经性质,随后被证明与Liouville算符有着紧密联系。其早期的相关工作主要集中于统计力学系统和非线性动力系统中Koopman算符谱性质的解析研究。Mezić 曾在2004 年的一篇文章中探讨了利用Koopman算符的某些本征函数对确定性以及随机非线性动力系统进行各态历经划分的理论和方法,然后于2005 年发表的文章中再次对Koopman 算符进行了谱分析的理论研究,并根据结论提出了可以利用其谱特征对高维动力系统(如流体系统和生物分子系统)进行模型简化的可能性。,2009 年Rowley、Mezić 等发表了第一篇有关Koopman 算符谱分析应用于流体力学中的文章,文章中介绍了基于Arnoldi 算法的Koopman 谱分析和模式分解方法:利用流体系统速度场的测量数据,构建Koopman 算符的矩阵表示,然后通过特征分解得到一系列本征值及对应的本征模式(Koopman 模式)。其中,每一个Koopman 模式都描述了非线性系统的一种动力学特征,相应本征值则描述了该模式的变化规律。Budisic等人在2012 年总结了Koopman 谱分析的理论和应用:在给定系统数据中,可以求得Koopman 算符的三种特性量,分别为本征值、本征函数和Koopman 模式。其中本征函数可以用于对系统进行各态历经的划分,Koopman 模式则用来简化模型以及研究不同变量之间的相关性。Steven等人在2017年提出了另一种Koopman分析的观点(HAVOK)并用SVD及时间延迟嵌入定理对常见的混沌系统做出分析,并预测了波峰出现的时机。

\section{本文主要内容及成果}





%% 本章参考文献
% \ifx\usechapbib\empty
% \nocite{BSTcontrol}
% \setcounter{NAT@ctr}{0}
% \bibliographystyle{buptgraduatethesis}
% \bibliography{references}
% \fi
