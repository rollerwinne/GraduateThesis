\chapter{引言}

当今时代是信息时代,社会的进步与发展充斥着大量的信息,包括能源、化工、生物、机械、交通等各项领域,然而在信息传递中愈发庞大的信息量中,如何有效的提取关键的信息是一个亟待解决的问题。

信息的具体表现形式是数据,随着数学与计算机等领域的发展,测量技术和计算能力的巨大改进,数据的提取与分析已成为一种新兴的学科应用在各个领域,然而数据分析的方法多种多样,不同的数据分析的方法有效性也不同,很多数据的分析方法都依赖于数据的数学特征。我们寄希望于找到一个普遍的数据分析方法,通过研究数据从而了解其内部的抽象结构。

在许多实际系统中,我们通常可以得到一系列关于该系统的丰富数据,然而由于不了解其动力学原理或者系统过于复杂等原因无法用比较准确的动力学方程来近似描述。为了了解和研究这些系统,通常进行大量的实验或观测得到系统的特性数据。尽管这些系统数据可以用现代的智能“黑盒子”例如各种神经网络\cite{hassoun1995fundamentals}来处理,但仍需要对系统的关键动力学特征有更深入的了解。我们希望能够找到一种方法,仅通过系统数据的演化过程得到系统的一些演化特征,并在这些演化特征中提取出关键特征。如果我们能提取出复杂系统的关键特征,就能通过分析实际复杂系统的长期行为,并对系统的行为作出一定程度的预测。在我们的现实生活中,若将各种事件抽象为一种数据,并能提取出关键特征并进行预测,将具有很大的现实意义。

\section{Koopman算符与动力系统的特征提取}

我们希望能够找到一种方法,仅通过系统数据的演化过程得到系统的一些演化特征,并在这些演化特征中提取出关键特征。Koopman算符给我们提供了一个有效的数学工具。Koopman算符由B.O.Koopman与1931年引入\cite{koopman1931hamiltonian},它作用在某个函数上,描述了函数的演化。若将系统的特性数据演化视为函数的演化,我们即可以用Koopman算符分析系统的特征。

Koopman分析方法仅由系统的特性数据得到,理论上在数据足够多的情况下,我们可以得到系统的一切特征,如何提取出关键的系统特征(或者说是我们想要的系统特征)也是一个很复杂的问题;另一方面,我们在实际应用中的数据总是有限的,也会存在一定的误差,于是复杂系统的Koopman分析依赖于大量的系统数据支持。为了简化计算复杂度,而又不失Koopman分析的核心思想,我们可以先选取某些已知的动力学系统作为Koopman分析的对象。

线性的动力学系统通过经典的动力学理论已经能够得到较好的解析分析方法,而非线性动力学系统的分析方法也有诸多理论用于系统的分析\cite{strogatz2001nonlinear},比如通过不动点的线性化分析得到不动点近邻区域的系统演化特征。而对于复杂的非线性动力学系统比如混沌系统,传统的非线性动力学分析方法有其局限性\cite{holmes1996coherent}。我们试图将Koopman分析应用在一些常见的混沌系统上,提取出混沌系统的系统特征,如果能抓住混沌系统的主要特征并进行定量的分析,就可以分析出混沌系统的长期行为,从而对混沌系统做一定程度上的预测。

混沌系统按照系统方程可以分为混沌映射(离散混沌)系统与混沌流(连续混沌)系统,其本质是相同的。但由于在分析过程中需要用到数值计算,而数值计算对混沌映射与混沌流的影响是不同的:混沌映射的数值计算误差来源主要依赖于计算机的精度误差;混沌流的数值计算误差来源除了计算机的精度误差,其主要误差来源于系统时间间隔的选取。另一方面,混沌流的数值计算过程也可以转变为一个关于系统时间间隔的映射系统\cite{strogatz2001nonlinear}。因此,我们的Koopman分析方法以混沌映射系统为基础,分别讨论混沌映射系统与混沌流系统。

混沌系统按照系统维度可以分为一维、二维、三维与高维系统,不同维度的混沌系统在计算方法与计算复杂度上有一定区别。有一些常见的混沌系统,比如Logistic映射\cite{phatak1995logistic}(一维映射),帐篷映射\cite{yoshida1983analytic}(一维映射),H\'{e}non映射\cite{henon1976two}(二维映射),Lorenz系统\cite{lorenz1963deterministic}(三维流),我们将选取几个典型的混沌系统作为分析对象,以验证Koopman分析的有效性。通过对上述一些常见的混沌系统的Koopman分析,我们可以构造出每个系统下的数据矩阵,进一步计算出Koopman算符的矩阵表示以及本征值、本征函数。Koopman算符的每个本征函数都代表了该系统的一种动力学模式,我们可以得到每个系统的很多本征函数,且本征函数的数量与相空间的数据点的数量相同,然而很多本征函数都是高度抽象的,很多本征函数反映了相同的动力学模式,我们的目的就是提取出每个系统不同的动力学模式。

在混沌系统中的Koopman分析可以为我们在以后实际应用中更复杂的非线性系统打下基础。通过已知的混沌系统的Koopman分析方式,我们可以观察到计算Koopman算符的本征函数时,不同的参数对本征函数的影响,比如基函数的形式、基函数的数量、基函数的分布、演化格点的分布等。在了解不同的参数对混沌系统的影响后,我们可以将Koopman分析方法应用到实际的复杂系统中。在实际的复杂系统中,我们可以采用与混沌系统同样的分析方法,即使系统非常复杂,我们也无需知道系统的工作内部机制以及动力学方程,只需要得到系统的状态演化过程,便能对复杂系统进行Koopman分析,并提取复杂系统的关键特征,从而分析复杂系统的长期行为,并对某些系统行为做出预测。

\section{国内外的研究现状}
关于数据分析与数据预测的方法有诸多理论。在本研究课题中采取Koopman算符进行数据的分析与预测。Koopman算符由B. O. Koopman于1931年提出\cite{koopman1931hamiltonian},它描述了动力学系统相空间中定义的函数的演化,与刘维尔算符密切相关\cite{gaspard1995spectral,gaspard2001liouvillian}。Koopman算符的本征值和本征函数可以描述系统动力学的全局特征\cite{budivsic2012applied},Mezi\'{c}等人首先将其用于相空间划分\cite{mezic2004comparison}以及简化高维动力学\cite{mezic2005spectral}的可能性。近年来,基于Koopman算符的动力学模式分解(DMD)算法及其改进版本已在不同场合进行了广泛测试,包括电力系统\cite{susuki2011nonlinear,susuki2012nonlinear}、建筑能源效率\cite{eisenhower2010decomposing,georgescu2012creating}和流体系统\cite{schmid2011applications,bagheri2013koopman,mezic2013analysis}等。DMD算法\cite{schmid2010dynamic}原则上可以根据测量数据通过严格的收敛性来计算频谱的精细结构\cite{korda2020data}。此外通过合理的安排数据序列可以构造Hankel矩阵\cite{arbabi2017ergodic},从而更方便的计算谱特征,此外其他更多的研究也揭示了线性化变化与Koopman算符谱之间的关系\cite{lan2013linearization}。

\section{本文主要内容及成果}
本文介绍了动力学系统与混沌系统的动力学,并利用Koopman算符分析一些常见混沌系统的动力学,在低维系统对动力学系统的相空间进行划分。Koopman算符通过计算本征值与本征函数能与符号动力学的边界点获得一致的划分,且可以通过控制基函数数量的精度来减小误差。这种方法为符号动力学和Koopman算符的本征函数建立了联系,并且这种寻找符号动力学边界的方法相较于计算集合拓扑结构更容易计算和实现,也更容易推广到一般的非线性系统。

在第二章中首先在2.1节介绍了动力学系统的基础知识,动力学系统包含线性动力学系统和非线性动力学系统;2.2节介绍了一种特殊的非线性动力学系统--混沌系统;2.3节介绍了Koopman算符的基本知识,包括Koopman算符的定义、本征值与本征函数、Koopman算符谱与Koopman算符的函数空间;2.4节将Koopman算符与动力学系统结合,介绍了符号动力学的边界点与动力学系统的线性化与周期轨道,说明了Koopman算符与动力学系统的联系。

在第三章中主要介绍了一些常见的混沌系统中Koopman算符的应用:在一维系统中的帐篷映射、Logistic映射,二维系统中的H\'{e}non映射,三维系统中的洛伦兹系统。在每个系统中,我们通过分析动力学系统的动力学,并计算Koopman算符的本征值和本征函数,来探究其之间的对应关系,以寻找出Koopman算符对动力学系统的动力学划分,我们还对低维系统中Koopman分析的鲁棒性、准确性和普适性进行了进一步的讨论,最后分析了Koopman算符在分析动力学系统模式与相空间划分的有效性。

最后第四章对本文进行了总结,并提出了Koopman算符的改进空间及展望。


%% 本章参考文献
% \ifx\usechapbib\empty
% \nocite{BSTcontrol}
% \setcounter{NAT@ctr}{0}
% \bibliographystyle{buptgraduatethesis}
% \bibliography{references}
% \fi
