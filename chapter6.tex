\chapter{开题报告}

\section{立项依据}
\subsection{研究目的与意义}
当今时代是一个信息时代,社会的发展充斥着各种信息交换,包括机械、能源、交通、化工、生物、航空航天等各项领域,然而在信息传递中愈发庞大的信息量中,如何有效的提取关键的信息是一个亟待解决的问题。

信息的具体表现形式是数据,数据和信息直接是相互联系的,数据是反映客观事物属性的记录。数据经过加工处理之后,就成为信息,而信息需要经过数字化转变成数据才能存储和传输。因此,数据分析也成为当下的一个热点问题。

随着数学与计算机等领域的发展,测量技术和计算能力的巨大改进,数据分析已成为一门新兴的学科应用在各个领域,并将继续发展。然而数据分析的方法多种多样,不同的数据分析方法的有效性也不同,很多数据的分析方法都依赖于数据的数学特征。我们寄希望于找到一个普遍的数据分析方法,通过研究数据从而了解其内部的抽象结构。

在许多实际系统中,我们通常可以得到一系列关于该系统的丰富数据,然而由于不了解其动力学原理或者系统过于复杂等原因无法用比较准确的动力学方程来近似描述。为了了解和研究这些系统,通常进行大量的实验或观测得到系统的特性数据。我们希望能够找到一种方法,仅通过系统数据的演化过程得到系统的一些演化特征,并在这些演化特征中提取出关键特征。如果我们能提取出复杂系统的关键特征,就能通过分析实际复杂系统的长期行为,并对系统的行为作出一定程度的预测。在我们的现实生活中,若将各种事件抽象为一种数据,并能提取出关键特征并进行预测,将具有很大的现实意义。

Koopman算符给我们提供了一个有效的数学工具。Koopman算法由B.O.Koopman与1931年引入,它作用在某个函数上,描述了函数的演化。若将系统的特性数据演化视为函数的演化,我们即可以用Koopman算符分析系统的特征。我们将用Koopman分析方法分析数据并作出一定程度的预测。

\subsection{国内外的研究现状}
关于数据分析与数据预测的方法有诸多理论。在本研究课题中采取Koopman算符进行数据的分析与预测。Koopman 算符由B. O. Koopman 于1931 年提出,能够描述系统可观察量随时间的演化,并可用来研究动力系统的各态历经性质,随后被证明与Liouville算符有着紧密联系。其早期的相关工作主要集中于统计力学系统和非线性动力系统中Koopman算符谱性质的解析研究。Mezić 曾在2004 年的一篇文章中探讨了利用Koopman算符的某些本征函数对确定性以及随机非线性动力系统进行各态历经划分的理论和方法,然后于2005 年发表的文章中再次对Koopman 算符进行了谱分析的理论研究,并根据结论提出了可以利用其谱特征对高维动力系统(如流体系统和生物分子系统)进行模型简化的可能性。,2009 年Rowley、Mezić 等发表了第一篇有关Koopman 算符谱分析应用于流体力学中的文章,文章中介绍了基于Arnoldi 算法的Koopman 谱分析和模式分解方法:利用流体系统速度场的测量数据,构建Koopman 算符的矩阵表示,然后通过特征分解得到一系列本征值及对应的本征模式(Koopman 模式)。其中,每一个Koopman 模式都描述了非线性系统的一种动力学特征,相应本征值则描述了该模式的变化规律。Budisic等人在2012 年总结了Koopman 谱分析的理论和应用:在给定系统数据中,可以求得Koopman 算符的三种特性量,分别为本征值、本征函数和Koopman 模式。其中本征函数可以用于对系统进行各态历经的划分,Koopman 模式则用来简化模型以及研究不同变量之间的相关性。Steven等人在2017年提出了另一种Koopman分析的观点(HAVOK)并用SVD及时间延迟嵌入定理对常见的混沌系统做出分析,并预测了波峰出现的时机。

\subsection{参考文献}
[1] 贾继莹. Koopman算符在一些动力系统中的算法和应用研究[D]. 2015.
[2] Brunton S L, Brunton B W, Proctor J L, et al. Chaos as an intermittently forced linear system[J]. Nature Communications, 2017, 8(1):19.
[3] Strogatz S H. From Kuramoto to Crawford: exploring the onset of synchronization in populations of coupled oscillators[J]. Physica D, 2000, 143(1-4):1-20.
[4] Hobson D. An Efficient Method for Computing Invariant Manifolds of Planar Maps[J]. Journal of Computational Physics, 1993, 104(1):14-22.
[5] Carles Simó. On the Hénon-Pomeau attractor[J]. Journal of Statistical Physics, 1979, 21(4):465-494.
[6] Pomeau Y, Berre M L. Dynamics of self-gravitating systems : Variations on a theme by Michel Henon[J]. Physics, 2014.
[7] Gaspard, P. Nicolis, G. Provata, A. Tasaki, S. Spectral signature of the pitchfork bifurcation: Liouville equation. Physical Review E, 1995.
[8] Gaspard P, Tasaki S. Liouvillian dynamics of the Hopf bifurcation[J]. Physical Review E, 2001, 64(5):056232.
[9] Korda M, Putinar M, Mezić, Igor. Data-driven spectral analysis of the Koopman operator[J]. 2017.
[10] Lan Y, Cvitanović P. Variational method for finding periodic orbits in a general flow[J]. Physical Review E, 2004, 69(1):016217.
[11] Steven H. Strogatz. Nonlinear dynamics and chaos: with applications to physics, biology, chemistry, and engineering [M]. Perseus Books Publishing, 2000
[12] Igor Mezić. Spectral Koopman operator in dynamical systems. Springer, 2017

\section{研究内容和目标}
\subsection{研究内容}
我们希望能够找到一种方法,仅通过系统数据的演化过程得到系统的一些演化特征,并在这些演化特征中提取出关键特征。Koopman算符给我们提供了一个有效的数学工具。Koopman算法由B.O.Koopman与1931年引入,它作用在某个函数上,描述了函数的演化。若将系统的特性数据演化视为函数的演化,我们即可以用Koopman算符分析系统的特征。

Koopman分析方法仅由系统的特性数据得到,理论上在数据足够多的情况下,我们可以得到系统的一切特征,如何提取出关键的系统特征(或者说是我们想要的系统特征)也是一个很复杂的问题;另一方面,我们在实际应用中的数据总是有限的,也会存在一定的误差,于是复杂系统的Koopman分析依赖于大量的系统数据支持。为了简化计算复杂度,而又不失Koopman分析的核心思想,我们可以先选取某些已知的动力学系统作为Koopman分析的对象。

线性的动力学系统通过经典的动力学理论已经能够得到较好的解析分析方法,而非线性动力学系统的分析方法也有诸多理论用于系统的分析,比如通过不动点的线性化分析得到不动点近邻区域的系统演化特征。而对于复杂的非线性动力学系统比如混沌系统,传统的非线性动力学分析方法有其局限性。我们试图将Koopman分析应用在某些混沌系统上,提取出混沌系统的系统特征,如果能抓住混沌系统的主要特征并进行定量的分析,就可以分析出混沌系统的长期行为,从而对混沌系统做一定程度上的预测。

混沌系统按照系统方程可以分为混沌映射(离散混沌系统)与混沌流(连续混沌系统),其本质是相同的。但由于在分析过程中需要用到数值计算,而数值计算对混沌映射与混沌流的影响是不同的:混沌映射的数值计算误差来源主要依赖于计算机的精度误差;混沌流的数值计算误差来源除了计算机的精度误差,其主要误差来源于系统时间间隔的选取。另一方面,混沌流的数值计算过程也可以转变为一个关于系统时间间隔的映射系统。因此,我们的Koopman分析方法以混沌映射系统为基础,分别讨论混沌映射系统与混沌流系统。

混沌系统按照系统维度可以分为一维、二维、三维与高维系统,不同维度的混沌系统在计算方法与计算复杂度上有一定区别。有一些常见的混沌系统,比如Logistic映射(一维映射),Henon映射(二维映射),Standard映射(二维映射),Lorenz系统(三维流),我们将选取几个典型的混沌系统作为分析对象,以验证Koopman分析的有效性。

Koopman算符

考虑在相空间$\mathbf{P}$上演化的动力学系统:对于$\mathbf{x}_p\in P$,$\mathbf{x}_{p+1}=T(\mathbf{x}_p)$。其中$p$为时间因子,$T$是$\mathbf{P}$到自身的映射,描述了相空间的演化规律。

定义作用在相空间函数$f(\mathbf{x})$上的Koopman算符$U$, 使得:
$$Uf\left( x\right)=f\left(T\left( x\right)\right)$$

Koopman算符描述了可观测量$g$的值沿相空间轨道的演化,若可观测量$g$在初始时刻的值为$g(x_0)$,则演化$p$步后可观测量$g$的值变为$g(x_p)=U^pg(x_0)$。

给定$g(x)$,若令函数$\tilde{g}(x)=g(T(x))$,则Koopman算符的作用为
$$Ug(x)=\tilde{g}(x)$$

因此Koopman算符的作用既可以看作映射$T$下$g(x)$函数值的演化,也可以看作函数形式的演化($g(x)\rightarrow\tilde{g}(x)$)。

Koopman算符U是线性算符,即使动力系统本身是非线性系统。即:
                $$U(\alpha g_1(x)+\beta g_2(x))=\alpha Ug_1(x)+\beta Ug_2(x)$$

其中$g_1(x)$,$g_2(x)$为相空间$P$上的任意标量函数,$α,β\in C$为任意标量。$U$是无穷维的算符,因此通过分析无穷维的线性算符$U$,能够体现有限维非线性系统的动力学特征。

许多实际的系统中,我们通常关注具有某些特征或意义的可观测量,这些量是定义在相空间的函数,由于其随时间变化的特征,它们通常是相应系统中Koopman算符的本征函数。Koopman曾用Koopman算符的谱性质研究动力系统的遍历性质。

Koopman算符$U$是一个线性算符,可以对其进行谱分解,设一个复标量函数$\phi_k(x)$为Koopman算符的本征函数,其相应的本征值为$\lambda_k$,则有
$$U\phi_k(x)=\phi_k(T(x))=\lambda_k\phi_k(x), k=1,2,\cdots$$

若$\phi_{k_1}(x)$,$\phi_{k_2}(x)$分别是Koopman算符$U$相应于本征值$\lambda_{k_1}$,$\lambda_{k_2}$的本征函数,则$\psi(x)=\phi_{k_1}(x)\phi_{k_2}(x)$也是$U$的本征函数,相应本征值为$λ_{k_1}λ_{k_2}$: 
$$U\psi(x)=\psi(T(x))=\phi_{k_1}(T(x))\phi_{k_2}(T(x))=\lambda_{k_1}\lambda_{k_2}\phi_{k_1}(x)φ_{k_2}(x)=\lambda_{k_1}\lambda_{k_2}\psi(x)$$

特别的,若$\lambda_k$是$U$的本征值,相应本征函数为$\phi_k(x)$,则$\lambda_k$的$n$次方${\lambda_k}^n$($n$为整数)也是$U$的本征值,相应本征函数为$\phi_k^n(x)=(\phi_k(x))^n$:
      $$U\phi_k^n(x)=(\phi_k(T(x)))^n=(\lambda_k\phi_k(x))^n=λ_k^n\phi_k^n(x), n\in Z, k=1,2,\cdots$$

两类特殊的本征值和本征函数:

1.当本征值为$\lambda_k=1$时,设相应本征函数为$\phi_k(x)$,则$\phi_k(x_p)=U\phi_k(x_{p-1})=\phi_k (x_{p-1})=\cdots =\phi_k(x_0)$($p\in Z$为时间指标),即沿着一条轨道本征函数值不随时间变化,这与物理系统中的守恒量相对应。

2.当本征值为$\lambda_k=e^{i\theta}$($\theta$为实数)时,设相应的本征函数为$\phi_k(x)$,则$\phi_k(x_p)=U\phi_k(x_{p-1})=e^{i\theta}\phi_k(x_{p-1})=\cdots=e^{ip\theta}\phi_k(x_0)$,即沿着一条轨道本征函数值模不变,而相位在变化。

Koopman算符的谱中模为1的本征值$|\lambda_k|=1$对应的本征函数的性质为在动力系统相空间寻找不变集和周期结构提供了一种标准:计算相空间中给定所有点在这些本征函数上的值,则本征函数值(或本征函数值的模)相同的点属于相空间的一个不变集(但并不一定是最小不变集)。

为了得到Koopman算符的本征函数,我们通常采用构造数据矩阵的方式,设某$p$时刻的数据为$\{x_{p_1},x_{p_2},\cdots,x_{p_n}\}$,$x_i\in \mathbb{R}^n$,当时刻为$p+1$时,数据演化为$\{x_{p_1+1},x_{p_2+1},\cdots,x_{p_n+1}\}$,$x_i\in \mathbb{R}^n$,我们可以选取一组基函数${g_j(x)}$,$j=1,2,\cdots,m$后,利用已知数据点将m个基函数及演化后的函数表示为在$\{x_{p_1},x_{p_2},\cdots,x_{p_n}\}$(称之为“演化格点”)下的列向量,从而构成两个数据矩阵:
\begin{gather}
    K=(g_1(x),g_2(x),\cdots,g_m(x))=
    \begin{pmatrix}
    g_1(x_{p_1}) & g_2(x_{p_1}) & \cdots & g_m(x_{p_1}) \\
    g_1(x_{p_2}) & g_2(x_{p_2}) & \cdots & g_m(x_{p_2}) \\
    \vdots       & \vdots       & \ddots & \vdots \\
    g_1(x_{p_n}) & g_2(x_{p_n}) & \cdots & g_m(x_{p_n})
    \end{pmatrix}
\end{gather}
\begin{gather}
    L=(\tilde{g}_1(x),\tilde{g}_2(x),\cdots,\tilde{g}_m(x))=
    \begin{pmatrix}
    g_1(x_{p_1+1}) & g_2(x_{p_1+1}) & \cdots & g_m(x_{p_1+1}) \\
    g_1(x_{p_2+1}) & g_2(x_{p_2+1}) & \cdots & g_m(x_{p_2+1}) \\
    \vdots         & \vdots         & \ddots & \vdots \\
    g_1(x_{p_n+1}) & g_2(x_{p_n+1}) & \cdots & g_m(x_{p_n+1})
    \end{pmatrix}
\end{gather}

其中$K$我们称为演化前矩阵,$L$称为演化后矩阵,$K$、$L$的每一列为相空间在一组点在某个基函数上的取值,当相空间的点数足够多时,可以视矩阵的每一列为一个基函数的离散数据表示,$K$、$L$则可以看作是多个基函数的组合,分别表示演化前的基函数和演化后的基函数,而演化后的基函数又可以看作一个新的函数,视为Koopman算符作用在基函数上得到。于是$K$、$L$之间的关系可由Koopman算符表示
$$UK = L$$

Koopman算符的矩阵表示可由上式确定,因此我们可以通过上式求得Koopman算符的矩阵表示,进一步求得Koopman算符的本征值与本征函数。

常见的混沌系统 

Logistic映射来源于生态学中的虫口模型,其动力学方程可描述为
\begin{equation}
    x_{n+1}=f(x_n)=1-2|x-\frac{1}{2}| ,\ x_n\in [0,1], n=1,2,3,\cdots
\end{equation}

\begin{equation}
    x_{n+1}=f(x_n)=\gamma x_n(1-x_n),\ x_n\in [0,1], n=1,2,3,\cdots
\end{equation}

通过非线性动力学的知识可以得到,该系统在$\gamma$取不同值时表现出不同的动力学行为。当$0<\gamma<1$时,系统最终都会渐进的趋于0;当$1<\gamma<3$时,系统会收敛到一个不动点,该不动点的值为$x^*=(\gamma-1)/\gamma$,此时系统的极限行为会趋于该不动点的值;当$3<\gamma<3.57$时,系统的迭代会出现周期行为,随着r的增大,周期的长度也会相应的增加,例如2周期,4周期,8周期等;当$3.57<\gamma<4$时,系统的迭代会在周期类型和混沌类型之间来回切换;直到$\gamma=4$时,系统处于完全混沌的状态。在我们的Koopman分析中,我们取$\gamma=4$的一个特例,通过Logistic映射的动力学方程演化出一系列的数据,作为Koopman分析的源数据,以此来分析Logistic映射的系统特征。

Henon映射是由法国数学家Michel H\'{e}non提出的,以此作为Lorenz模型的Poincar\'{e}界面的简化模型。Henon映射是一个可以产生混沌现象的离散时间动态系统,其迭代方程可以描述为

\begin{equation}
    \begin{cases}
        x_{n+1}=y_n+1-ax_n^2\\
        y_{n+1}=bx_n
    \end{cases}\ x,y\in [-1.5,1.5]
\end{equation}

在经典的Henon映射中,我们通常取$a=1.4$与$b=0.3$,此时系统表现出混沌现象。在我们的Koopman分析中,我们在取上述参数值的情况下,通过Henon映射的迭代方程演化出一系列的数据,作为Koopman分析的源数据,以此来分析Henon映射的系统特征。

Lorenz系统是1963年由Edward Lorenz提出的描述空气流体运动的一个数学模型,通常用Lorenz方程来描述
\begin{equation}
    \begin{cases}
        \dot{x}=\sigma(y-x)\\
        \dot{y}=x(\rho-z)-y\\
        \dot{z}=xy-\beta z
    \end{cases}\
\end{equation}

该方程描述了一个三维相空间的三个分量对时间的变化率,是一个连续时间动态系统。Lorenz系统具有非线性、非周期性和确定性的性质。参数$\sigma$,$\rho$,$\beta$取不同值时Lorenz系统表现出不同的动力学行为。如在$\rho<1$时,系统只有一个不动点,即原点,所有轨道的长期行为都趋于原点;当$\rho=1$时系统发生了叉式分叉,在$\rho>1$时出现了两个不动点,不动点的稳定性可由其他参数满足的条件确定;当我们将三个参数取一组特定的值$\rho=28$,$\sigma=10$,$\beta=8/3$时,Lorenz方程的解是混沌的,相空间会存在两个奇异吸引子。Lorenz系统的这种动力学特征用传统的非线性动力学知识是比较难分析的,我们将在取上述参数的条件下,通过Lorenz微分方程演化出相空间的一条轨道,作为Koopman分析的源数据,以此来分析Lorenz系统的特征。


\subsection{研究目标和效果}

通过对上述一些常见的混沌系统的Koopman分析,我们可以构造出每个系统下的数据矩阵,进一步计算出Koopman算符的矩阵表示以及本征值、本征函数。每个本征函数都代表了该系统的一种动力学模式,我们可以得到每个系统的很多本征函数,且本征函数的数量与相空间的数据点的数量相同,然而很多本征函数都是高度抽象的,很多本征函数反映了相同的动力学模式,而我们的工作就是提取出每个系统不同的动力学模式。

在混沌系统中的Koopman分析可以为我们在以后实际应用中更复杂的非线性系统打下基础。通过已知的混沌系统的Koopman分析方式,我们可以观察到计算Koopman算符的本征函数时,不同的参数对本征函数的影响,比如基函数的形式、基函数的数量、基函数的分布、演化格点的分布等。在了解不同的参数对混沌系统的影响后,我们可以将Koopman分析方法应用到实际的复杂系统中。在实际的复杂系统中,我们可以采用与混沌系统同样的分析方法,即使系统非常复杂,我们也无需知道系统的工作内部机制以及动力学方程,只需要得到系统的状态演化过程,便能对复杂系统进行Koopman分析,并提取复杂系统的关键特征,从而分析复杂系统的长期行为,并对某些系统行为做出预测。

\section{研究方案设计及可行性分析}
\subsection{研究方案设计与技术路线}
以Logistic映射为例,我们介绍Koopman分析的算法实现与具体步骤,不失一般性,我们取特定的参数$\gamma=4$,此时Logistic映射表现为混沌状态,此时Logistic映射的动力学方程可以描述为
$$x_{n+1}=f(x_n)=4x_n(1-x_n),x_n\in [0,1],n=1,2,3,\cdots$$

该系统为一维系统,相空间范围为$[0,1]$,该系统存在两个不动点:$0$和$3/4$。我们首先根据该映射方程产生一组迭代数据,设我们产生的迭代数据的数量为$n+1$。则每个数据点可以表示为$x_i,i=1,2,\cdots,n+1$,为了得到“演化前数据”和“演化后数据”,我们将这组迭代产生的数据构造为$\{x_1,x_2,\cdots,x_n\}$与$\{x_2,x_3,\cdots,x_{n+1}\}$,并根据这两组数据构造出数据矩阵$K$与$L$。

在构造矩阵$K$与$L$时,我们需要选定一组基函数${g_j(x)},j=1,2,\cdots,m$,基函数的选取对我们计算本征函数的结果至关重要,于是我们将从多个角度讨论基函数的选取对本征函数的影响。基函数的选取至关重要,为了能较好的表示出Koopman算符的本征函数,需要一组较完备的基函数作为算符作用的可观测量,可以选取Gauss基函数、Fourier基函数、Legendre基函数。然而上述基函数只有在数量趋于无穷时才具有完备性,我们希望基函数的数量足够多,然而为了相空间的数据点能够较好的表示每个基函数,基函数的数量又不宜超过演化格点的数量,此外考虑到计算量的因素,我们只能将基函数的数量设为某个有限值。此外由于相空间粒子分布的不均匀性,我们还可以取不同分布的基函数。总之,基函数的选取有多重形式,而基函数的选取会影响到本征函数的计算,因此基函数的选取至关重要。

\subsection{理论分析与计算}
在选取合适的基函数与构造数据矩阵K与L后,我们即可以根据$UK=L$计算求得Koopman算符的矩阵表示以及本征值与本征函数,通常我们会得到n个本征值与本征函数,而在这些本征值中,我们比较关心的是本征值接近1的本征函数,因为本征值为1的本征函数反映了相空间中一条随时间演化不变的轨道,这种轨道即是系统的一个关键特征,而由于我们的数值计算误差,因此我们将选取接近1的本征值对应的本征函数。

下图给出了一个基函数取Fourier基函数$n=1000$,$m=200$时的9个本征函数图像,每个图像的标题表示该本征函数对应的本征值:

\subsection{可行性分析}
Koopman分析的可行性可以通过混沌系统的已知特性来得到验证,如在Logistic映射中有两个不动点,而我们得到的本征函数图像恰好在不动点以及不动点的原像处得到了本征函数的极值点,说明了本征函数能够体现系统的动力学特征;此外,我们还可以通过验证动力学系统的周期点与周期轨道等特征,对比Koopman算符的本征函数,以此观察Koopman算符分析动力学系统的可行性。

\section{本研究课题可能的创新之处}
\begin{description}
    \item[给出了一个分析动力学系统的普适性方法]

    线性的动力学系统通过经典的动力学理论已经能够得到较好的解析分析方法,而非线性动力学系统的分析方法也有诸多理论用于系统的分析,比如通过不动点的线性化分析得到不动点近邻区域的系统演化特征。而对于复杂的非线性动力学系统比如混沌系统,传统的非线性动力学分析方法有其局限性。目前有许多分析非线性系统的行为的定性技巧,但是这些技巧没有一个对所有非线性系统都适用,大多数的技巧都只适用与一些特殊的情形。而我们提出的Koopman分析方法的理论和算法适用所有的动力学系统,因此在很多系统上都具有极其广泛的应用前景。

    \item[从函数演化的角度分析常见的混沌系统并验证Koopman分析的有效性]
     
    对于常见的一些混沌系统,其动力学特征已被很多前人研究过,我们使用Koopman分析方法分析其动力学特征去验证我们分析方法的正确性,同时给出了一种新的分析动力学系统的Koopman分析方法,即通过函数演化的角度去分析动力学特征。

    \item[给出了对于一般复杂系统的分析方法]
     
    通过Koopman分析在常见混沌系统中的特征提取,我们给出了对于一般复杂系统的分析方式,即使我们不知道系统的工作内部机制以及动力学方程,只需要得到系统的状态演化过程,便能对复杂系统进行定性的分析。由于其普适性,我们可以将Koopman分析应用到各种复杂问题上,如生物大分子的运动、流体系统、模式识别以及气候变化等。
 
\end{description}

\section{研究基础与工作条件}
\subsection{研究基础}
前人研究基础:Koopman 算符由B. O. Koopman 于1931 年提出,能够描述系统可观察量随时间的演化,并可用来研究动力系统的各态历经性质,随后被证明与Liouville算符有着紧密联系。其早期的相关工作主要集中于统计力学系统和非线性动力系统中Koopman算符谱性质的解析研究。Mezić 曾在2004 年的一篇文章中探讨了利用Koopman算符的某些本征函数对确定性以及随机非线性动力系统进行各态历经划分的理论和方法,然后于2005 年发表的文章中再次对Koopman 算符进行了谱分析的理论研究,并根据结论提出了可以利用其谱特征对高维动力系统(如流体系统和生物分子系统)进行模型简化的可能性。2009 年Rowley、Mezić 等发表了第一篇有关Koopman 算符谱分析应用于流体力学中的文章,文章中介绍了基于Arnoldi 算法的Koopman 谱分析和模式分解方法:利用流体系统速度场的测量数据,构建Koopman 算符的矩阵表示,然后通过特征分解得到一系列本征值及对应的本征模式(Koopman 模式)。其中,每一个Koopman 模式都描述了非线性系统的一种动力学特征,相应本征值则描述了该模式的变化规律。Steven等人在2017年提出了另一种Koopman分析的观点(HAVOK)并用SVD及时间延迟嵌入定理对常见的混沌系统做出分析,并预测了波峰出现的时机。

前置学习课程:非线性动力学、高等数学、线性代数、数学物理方法、矩阵论、信息论、数据结构。

计算机语言基础:MATLAB、Mathematica、Python、C。
\subsection{实验条件}
计算机仿真环境:
操作系统:Windows 7 Ultimate Service Pack 1 64bit
CPU:Intel(R) Core(TM) i5-6500 3.20GHz
内存:16GB DDR3
Matlab:MathWorks MATLAB R2018b
Mathematica:Wolfram Mathematica 11.3
Python:Python 3.7

\subsection{缺少的实验条件}
理论上Koopman分析方法需要无限多的源数据,而我们的数值计算中的数据总是有限的,但我们希望能尽可能多的利用源数据进行数值计算,这就需要更多的计算量,而实验室计算机难以计算足够大的矩阵,这为我们的仿真造成了一定程度的困难。

\subsection{拟解决途径}
从硬件角度我们可以通过更新硬件设备或者使用超级计算机进行仿真;从编程的角度我们可以通过更深入的学习程序设计使我们的程序运行效率达到更优,例如采用并行计算;从基础理论角度我们可以适当的优化理论算法,使我们能更有效的从源数据中提取出系统的关键特性。

%% 本章参考文献
% \ifx\usechapbib\empty
% \nocite{BSTcontrol}
% \setcounter{NAT@ctr}{0}
% \bibliographystyle{buptgraduatethesis}
% \bibliography{references}
% \fi
