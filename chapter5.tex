\chapter{总结与展望}
\section{总结}
Koopman 算符描述了相空间的函数的演化,我们可以用Koopman 算符分析系统的演化特征,寻找动力学系统的演化规律。通过对Koopman算符进行谱分解,我们可以得到其本征值与本征函数,我们关心$\lambda\rightarrow 1$的本征值及其对应的本征函数,由于我们基函数选取的总是不够完备,因此我们的本征函数可能会小于1。我们关心较大的本征值并将其本征函数描绘在相空间中,此时Koopman算符的本征函数存在一些极值点,通过我们先前的讨论,这些极值点对应着我们符号动力学边界点的划分。

基函数的不同,会导致Koopman算符在不同函数空间的表示的不同,高斯基函数是一组近似局部、近似完备的函数空间,我们可以将Koopman算符的本征函数在此空间中描述,但是由于我们的基函数数量总是有限的,而Koopman算符的本征函数作为无限维的函数,我们得到的本征函数值只是在此函数空间中的一个投影。但同时有限个基函数可以使我们一定可以观察到在每个基函数上的权重的不同,从而使我们对相空间的区域进行划分,这些划分的区域代表了相空间不同的动力学特征。

在自然基函数空间中,我们仅利用一组粒子的演化数据作为一组基函数,并将该组基函数演化$m$次来计算Koopman算符的本征函数。由于粒子的演化数据随着动力学方程分布,因此我们可以通过有限次迭代来反映动力学的演化规律。在动力学中我们每演化一次,我们的相空间就会发生一次变换(通常是拉伸再折叠的过程),Koopman算符的本征函数也反映了这种规律:基函数的数量$m$每增加1,Koopman算符的极值点数量近似增加一倍,对应的边界点的位置也增加了一倍,反映出边界点的层次也增加了一层。

Koopman算符可以用于任何其他的动力学系统。Koopman算符的本征值与本征函数为我们寻找符号动力学的边界点提供了一种计算方法:对于一个复杂系统,有时我们无法精确描述其动力学特征,我们可以通过符号动力学的划分对其进行粗粒度的描述。我们可以通过Koopman算符的本征值与本征函数寻找其极值点,来寻找该系统对应的符号动力学的分界点。且随着我们增加基函数的数量,我们便能更精细的刻画符号动力学的分界点。当我们基函数的数量取的足够多时,理论上我们可以寻找到其所有的边界点。若能找到动力学系统的边界点,我们便可以通过一组符号序列对相空间的不同区域进行描述,并划分出相空间,进一步讨论这些不同区域的动力学特征,通过符号动力学的符号序列,我们还可以预测系统的长期行为。由于其在复杂系统中的普适性,我们可以将Koopman分析应用到各种复杂问题上,如生物大分子的运动、流体系统、模式识别以及气候变化等。

\section{改进空间及展望}

Koopman算符描述了相空间可观测函数的演化,并可以找到重要的动力学模式。在Koopman算符的谱分解中,可以识别出许多本征值和本征函数,但是要找出最相关的本征值并解释其动力学意义仍然是一个巨大的挑战。在本文中,我们选取了适当条件的本征函数的极值,并与一些混沌系统中符号动力学的边界点位置作了一定的对比,发现它们在一定误差范围内重叠在一起。在我们选取的一定的基函数下,很容易可以计算出Koopman算符的谱,且精度随着基函数的数量增加而增加。与其他基函数相比,直接从演化数据中提取的自然演化的基函数是最有效的,且这种方法应适用于其他非线性与混沌系统。

在非线性系统的研究中,符号动力学的划分一直是一个关键的问题,尤其当相空间的维度是高维时。过去的方法都集中在构造稳定和不稳定的流形或确定某些拓扑指数上,这些特征在轨道结构比较复杂时很难计算,比如在耦合的非线性系统中。在本文中,通过计算Koopman算符的谱特征采取了完全不同的路线,从而避免了对相空间复杂的几何特征进行描述。因此,将我们的方法扩展到复杂的非线性系统中的处理并获取适当的符号动力学描述是合理的。

虽然Koopman算符对相空间的划分在一般的低维混沌系统的分析已得到证实,但我们仍需进一步的研究来扩展到更高的维度,尤其是在具有多个不稳定方向的非线性系统中。在本文中,选取本征值接近1的本征函数,我们观察该本征函数的特征,并巧妙的讨论了与之相关的本征函数的特征,我们比较了不同基函数数量情况下的本征函数,最终发现这些本征函数中的极值点分布是一致的,但是随着基函数的增加,出现了新的极值点,这对应了我们更高层级的边界点。因此,在分析动力学特征和划分相空间时,我们有必要通过调整基函数数量的大小来粗化或细化对相空间的描述,这也需要更详细的探讨。

在低维系统中,我们还将系统的演化过程引入了高斯白噪声,用以模拟现实中的涨落。但是Koopman算符的本征函数对相空间的划分与无噪声的情况相比并没有太大的区别,这表明我们的方法具有鲁棒性。相空间的划分是否随噪声强度变化以及如何随噪声强度变化也是一件值得讨论的事情,即理论上应存在一个有限大小的最佳分区。在高维混沌系统中,当我们选取自然演化的基函数时,如二维Henon映射中,我们选取x方向的演化用于计算Koopman算符的矩阵表示,因为在这个方向上有非线性项,对应几何空间中的拉伸和折叠。而在一般的系统中,我们并不能确定哪个方向上最重要,若能通过自动化的计算来确定最关键的维度特征,则系统程序的开发则更具有普适性。当然在实际系统中,通常我们只收集系统的部分数据,甚至可能不包含所需的数据,而从此类数据中能够提取多少信息以及能够提取出怎样的信息也具有重要的现实意义。

% 通过对常见混沌系统的Koopman分析,我们可以得到许多有用的结论:
% \begin{itemize}
%     \item 在Koopman算符中,本征函数值相等的点属于一个不变集。在动力学系统中,不变集与周期轨道密切相关。
%     \item Koopman算符本征函数值的“极值点”在某些情况下与系统的“边界点”非常吻合。“边界点”反映了动力学系统的符号动力学划分,而符号动力学的划分可以预测系统的长期行为。
%     \item 基函数的不同,会导致Koopman算符在不同函数空间的表示的不同,因此我们会看到不同的本征函数图像,但是在几乎所有的函数空间中,边界点都能较好的和系统的临界点对应,可见Koopman算符能较好的划分系统的符号动力学边界。自然基下相同。
% \end{itemize}

% Koopman算符对动力学系统分析的意义有一下几个方面:
% \begin{itemize}
%     \item 给出了一个分析动力学系统的普适性方法。线性的动力学系统通过经典的动力学理论已经能够得到较好的解析分析方法,而非线性动力学系统的分析方法也有诸多理论用于系统的分析,比如通过不动点的线性化分析得到不动点近邻区域的系统演化特征。而对于复杂的非线性动力学系统比如混沌系统,传统的非线性动力学分析方法有其局限性。目前有许多分析非线性系统的行为的定性技巧,但是这些技巧没有一个对所有非线性系统都适用,大多数的技巧都只适用与一些特殊的情形。而我们提出的Koopman分析方法的理论和算法适用所有的动力学系统,因此在很多系统上都具有极其广泛的应用前景。
%     \item 从函数演化的角度分析常见的混沌系统并验证Koopman分析的有效性。Koopman 算符提供了一个有效的数学工具,作用于某个函数并描述了该函数的演化。若将系统特征数据的演化看作为相空间的函数的演化,则可以用Koopman 算符分析系统的演化特征,进一步提取关键特性,并在一定程度上预测系统的长期行为。对于常见的一些混沌系统,其动力学特征已被很多前人研究过,我们使用Koopman分析方法分析其动力学特征去验证我们分析方法的正确性,同时给出了一种新的分析动力学系统的Koopman分析方法,即通过函数演化的角度去分析动力学特征。
%     \item 给出了对于一般复杂系统的分析方法。通过Koopman分析在常见混沌系统中的特征提取,我们给出了对于一般复杂系统的分析方式,即使我们不知道系统的工作内部机制以及动力学方程,只需要得到系统的状态演化过程,便能对复杂系统进行定性的分析。由于其普适性,我们可以将Koopman分析应用到各种复杂问题上,如生物大分子的运动、流体系统、模式识别以及气候变化等。
% \end{itemize}

%% 本章参考文献
% \ifx\usechapbib\empty
% \nocite{BSTcontrol}
% \setcounter{NAT@ctr}{0}
% \bibliographystyle{buptgraduatethesis}
% \bibliography{references}
% \fi
