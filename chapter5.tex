\chapter{总结与展望}
\section{问题及整改方案}
1.如何精确的用一些随机梯度下降算法求得精简的本征值。以及如何跳出局部最优解。
2.随着基函数的增多,本征函数的划分也变得更精确,如何通过粗粒度的划分来实现细粒度的划分。
3.如果原方程加了噪声,会对Koopman算符本征函数有何影响。
4.Koopman算符的本征函数是否具有鲁棒性。
5.如何确定边界点的层次。
6.能否用通用的方法画出动力学系统的分界线。
7.本征函数的最大值最小值分别表示什么含义。
8.基函数的变化是如何影响本征函数的。
以上都是在课题中遇到的问题,今后会继续加强对这些问题的思考与完善,并将这些思考过程择重点记录到论文中。
\section{本研究课题可能的创新之处}
\begin{description}
    \item[给出了一个分析动力学系统的普适性方法]

    线性的动力学系统通过经典的动力学理论已经能够得到较好的解析分析方法,而非线性动力学系统的分析方法也有诸多理论用于系统的分析,比如通过不动点的线性化分析得到不动点近邻区域的系统演化特征。而对于复杂的非线性动力学系统比如混沌系统,传统的非线性动力学分析方法有其局限性。目前有许多分析非线性系统的行为的定性技巧,但是这些技巧没有一个对所有非线性系统都适用,大多数的技巧都只适用与一些特殊的情形。而我们提出的Koopman分析方法的理论和算法适用所有的动力学系统,因此在很多系统上都具有极其广泛的应用前景。

    \item[从函数演化的角度分析常见的混沌系统并验证Koopman分析的有效性]
     
    对于常见的一些混沌系统,其动力学特征已被很多前人研究过,我们使用Koopman分析方法分析其动力学特征去验证我们分析方法的正确性,同时给出了一种新的分析动力学系统的Koopman分析方法,即通过函数演化的角度去分析动力学特征。

    \item[给出了对于一般复杂系统的分析方法]
     
    通过Koopman分析在常见混沌系统中的特征提取,我们给出了对于一般复杂系统的分析方式,即使我们不知道系统的工作内部机制以及动力学方程,只需要得到系统的状态演化过程,便能对复杂系统进行定性的分析。由于其普适性,我们可以将Koopman分析应用到各种复杂问题上,如生物大分子的运动、流体系统、模式识别以及气候变化等。
 
\end{description}
%% 本章参考文献
% \ifx\usechapbib\empty
% \nocite{BSTcontrol}
% \setcounter{NAT@ctr}{0}
% \bibliographystyle{buptgraduatethesis}
% \bibliography{references}
% \fi
