\chapter{中期报告}
\section{研究内容简介}
\subsection{选题背景}
当今时代是一个信息时代,社会的发展充斥着各种信息交换,包括机械、能源、交通、化工、生物、航空航天等各项领域,然而在信息传递中愈发庞大的信息量中,如何有效的提取关键的信息是一个亟待解决的问题。

信息的具体表现形式是数据,数据和信息直接是相互联系的,数据是反映客观事物属性的记录。数据经过加工处理之后,就成为信息,而信息需要经过数字化转变成数据才能存储和传输。因此,数据分析也成为当下的一个热点问题。

随着数学与计算机等领域的发展,测量技术和计算能力的巨大改进,数据分析已成为一门新兴的学科应用在各个领域,并将继续发展。然而数据分析的方法多种多样,不同的数据分析方法的有效性也不同,很多数据的分析方法都依赖于数据的数学特征。我们寄希望于找到一个普遍的数据分析方法,通过研究数据从而了解其内部的抽象结构。

在许多实际系统中,我们通常可以得到一系列关于该系统的丰富数据,然而由于不了解其动力学原理或者系统过于复杂等原因无法用比较准确的动力学方程来近似描述。为了了解和研究这些系统,通常进行大量的实验或观测得到系统的特性数据。我们希望能够找到一种方法,仅通过系统数据的演化过程得到系统的一些演化特征,并在这些演化特征中提取出关键特征。如果我们能提取出复杂系统的关键特征,就能通过分析实际复杂系统的长期行为,并对系统的行为作出一定程度的预测。在我们的现实生活中,若将各种事件抽象为一种数据,并能提取出关键特征并进行预测,将具有很大的现实意义。

\subsection{研究内容}
我们希望能够找到一种方法,仅通过系统数据的演化过程得到系统的一些演化特征,并在这些演化特征中提取出关键特征。Koopman算符给我们提供了一个有效的数学工具。Koopman算法由B.O.Koopman与1931年引入,它作用在某个函数上,描述了函数的演化。若将系统的特性数据演化视为函数的演化,我们即可以用Koopman算符分析系统的特征。

Koopman分析方法仅由系统的特性数据得到,理论上在数据足够多的情况下,我们可以得到系统的一切特征,如何提取出关键的系统特征(或者说是我们想要的系统特征)也是一个很复杂的问题;另一方面,我们在实际应用中的数据总是有限的,也会存在一定的误差,于是复杂系统的Koopman分析依赖于大量的系统数据支持。为了简化计算复杂度,而又不失Koopman分析的核心思想,我们可以先选取某些已知的动力学系统作为Koopman分析的对象。

线性的动力学系统通过经典的动力学理论已经能够得到较好的解析分析方法,而非线性动力学系统的分析方法也有诸多理论用于系统的分析,比如通过不动点的线性化分析得到不动点近邻区域的系统演化特征。而对于复杂的非线性动力学系统比如混沌系统,传统的非线性动力学分析方法有其局限性。我们试图将Koopman分析应用在某些混沌系统上,提取出混沌系统的系统特征,如果能抓住混沌系统的主要特征并进行定量的分析,就可以分析出混沌系统的长期行为,从而对混沌系统做一定程度上的预测。

混沌系统按照系统方程可以分为混沌映射(离散混沌系统)与混沌流(连续混沌系统),其本质是相同的。但由于在分析过程中需要用到数值计算,而数值计算对混沌映射与混沌流的影响是不同的:混沌映射的数值计算误差来源主要依赖于计算机的精度误差;混沌流的数值计算误差来源除了计算机的精度误差,其主要误差来源于系统时间间隔的选取。另一方面,混沌流的数值计算过程也可以转变为一个关于系统时间间隔的映射系统。因此,我们的Koopman分析方法以混沌映射系统为基础,分别讨论混沌映射系统与混沌流系统。

混沌系统按照系统维度可以分为一维、二维、三维与高维系统,不同维度的混沌系统在计算方法与计算复杂度上有一定区别。有一些常见的混沌系统,比如Logistic映射(一维映射),Henon映射(二维映射),Standard映射(二维映射),Lorenz系统(三维流),我们将选取几个典型的混沌系统作为分析对象,以验证Koopman分析的有效性。

\subsection{关键技术}
Koopman算符给我们提供了一个有效的数学工具。Koopman算法由B.O.Koopman与1931年引入,它作用在某个函数上,描述了函数的演化。若将系统的特性数据演化视为函数的演化,我们即可以用Koopman算符分析系统的特征。我们将用Koopman分析方法分析数据并作出一定程度的预测。
关于数据分析与数据预测的方法有诸多理论。在本研究课题中采取Koopman算符进行数据的分析与预测。Koopman 算符由B. O. Koopman 于1931 年提出,能够描述系统可观察量随时间的演化,并可用来研究动力系统的各态历经性质,随后被证明与Liouville算符有着紧密联系。其早期的相关工作主要集中于统计力学系统和非线性动力系统中Koopman算符谱性质的解析研究。Mezić 曾在2004 年的一篇文章中探讨了利用Koopman算符的某些本征函数对确定性以及随机非线性动力系统进行各态历经划分的理论和方法,然后于2005 年发表的文章中再次对Koopman 算符进行了谱分析的理论研究,并根据结论提出了可以利用其谱特征对高维动力系统(如流体系统和生物分子系统)进行模型简化的可能性。,2009 年Rowley、Mezić 等发表了第一篇有关Koopman 算符谱分析应用于流体力学中的文章,文章中介绍了基于Arnoldi 算法的Koopman 谱分析和模式分解方法:利用流体系统速度场的测量数据,构建Koopman 算符的矩阵表示,然后通过特征分解得到一系列本征值及对应的本征模式(Koopman 模式)。其中,每一个Koopman 模式都描述了非线性系统的一种动力学特征,相应本征值则描述了该模式的变化规律。Budisic等人在2012 年总结了Koopman 谱分析的理论和应用:在给定系统数据中,可以求得Koopman 算符的三种特性量,分别为本征值、本征函数和Koopman 模式。其中本征函数可以用于对系统进行各态历经的划分,Koopman 模式则用来简化模型以及研究不同变量之间的相关性。Steven等人在2017年提出了另一种Koopman分析的观点(HAVOK)并用SVD及时间延迟嵌入定理对常见的混沌系统做出分析,并预测了波峰出现的时机。

% 1.1 Koopman算符
% 考虑在相空间P上演化的离散动力系统:对于x_p∈P,x_(p+1)=T(x_p) 
% 定义作用在相空间函数g(x)上的Koopman 算符U, 使得:
% Ug(x)=g(T(x))
% Koopman算符描述了可观测量g的值沿相空间轨道的演化,若可观测量g在初始时刻的值g(x_0)为,则演化p步后可观测量g的值变为g(x_p )=U^p g(x_0)。
% 给定g(x),若令函数g ̃(x)=g(T(x)),则Koopman算符的作用为
% Ug(x)=g ̃(x)
% 因此Koopman算符的作用既可以看作映射T下g(x)函数值的演化,也可以看作函数形式的演化(g(x)→g ̃(x))。
% Koopman算符U是线性算符,即使动力系统本身是非线性系统。即:
% U(αg_1 (x)+βg_2 (x))=αUg_1 (x)+βUg_2 (x)
% 其中g_1 (x),g_2 (x)为相空间P上的任意标量函数,α,βϵC为任意标量。U是无穷维的算符,因此通过分析无穷维的线性算符U,能够体现有限维非线性系统的动力学特征。
% 许多实际的系统中,我们通常关注具有某些特征或意义的可观测量,这些量是定义在相空间的函数,由于其随时间变化的特征,它们通常是相应系统中Koopman算符的本征函数。Koopman曾用Koopman算符的谱性质研究动力系统的遍历性质。
% Koopman算符U是一个线性算符,可以对其进行谱分解,设一个复标量函数φ_k (x)为Koopman算符的本征函数,其相应的本征值为λ_k,则有
% Uφ_k (x)=φ_k (T(x))=λ_k φ_k (x),k=1,2,⋯
% 若φ_(k_1 ) (x),φ_(k_2 ) (x)分别是Koopman算符U相应于本征值λ_(k_1 ),λ_(k_2 )的本征函数,则ψ(x)=φ_(k_1 ) (x) φ_(k_2 ) (x)也是U的本征函数,相应本征值为λ_(k_1 ) λ_(k_2 ): 
% Uψ(x)=ψ(T(x))=φ_(k_1 ) (T(x)) φ_(k_2 ) (T(x))=λ_(k_1 ) λ_(k_2 ) φ_(k_1 ) (x) φ_(k_2 ) (x)=λ_(k_1 ) λ_(k_2 ) ψ(x)
% 特别的,若λ_k是U的本征值,相应本征函数为φ_k (x),则λ_k的n次方〖λ_k〗^n(n为整数)也是U的本征值,相应本征函数为〖φ_k〗^n (x)=(〖φ_k (x))〗^n:
%       U〖φ_k〗^n (x)=(〖φ_k (T(x)))〗^n=(〖λ_k φ_k (x)〗^n )=〖λ_k〗^n 〖φ_k〗^n (x),nϵZ,k=1,2,⋯
% 两类特殊的本征值和本征函数:
% 1.当本征值为λ_k=1时,设相应本征函数为φ_k (x),则φ_k (x_p )=Uφ_k (x_(p-1) )=φ_k (x_(p-1) )=⋯=φ_k (x_0 ) (pϵZ为时间指标),即沿着一条轨道本征函数值不随时间变化,这与物理系统中的守恒量相对应。
% 2.当本征值为λ_k=e^iθ(θ为实数)时,设相应的本征函数为φ_k (x),则φ_k (x_p )=Uφ_k (x_(p-1) )=e^iθ φ_k (x_(p-1) )=⋯=e^ipθ φ_k (x_0 ),即沿着一条轨道本征函数值模不变,而相位在变化。
% Koopman算符的谱中模为1的本征值|λ_k |=1对应的本征函数的性质为在动力系统相空间寻找不变集和周期结构提供了一种标准:计算相空间中给定所有点在这些本征函数上的值,则本征函数值(或本征函数值的模)相同的点属于相空间的一个不变集(但并不一定是最小不变集)。
% 为了得到Koopman算符的本征函数,我们通常采用构造数据矩阵的方式,设某p时刻的数据为{x_(p_1 ),x_(p_2,),⋯x_(p_n ) } ,x_i∈R^N,当时刻为p+1时,数据演化为{x_(p_1+1),x_(p_2+1,),⋯x_(p_n+1) } ,x_i∈R^N,我们可以选取一组基函数{g_j (x)},j=1,2,⋯m后,利用已知数据点将m个基函数及演化后的函数表示为在{x_(p_1 ),x_(p_2,)⋯x_(p_n ) }(称之为“演化格点”)下的列向量,从而构成两个数据矩阵:
% K=(g_1 (x),g_2 (x),⋯g_m (x))=(■(g_1 (x_(p_1 ) )&g_2 (x_(p_1 ) )  ⋯&g_m (x_(p_1 ))@⋮&⋱&⋮@g_1 (x_(p_n ) )&g_2 (x_(p_n ))⋯&g_m (x_(p_n ))))
% L=(g ̃_1 (x),g ̃_2 (x),⋯g ̃_m (x))= (■(g_1 (x_(p_1+1) )&g_2 (x_(p_1+1) )  ⋯&g_m (x_(p_1+1))@⋮&⋱&⋮@g_1 (x_(p_n+1) )&g_2 (x_(p_n+1))⋯&g_m (x_(p_n+1))))
% 其中K我们称为演化前矩阵,L称为演化后矩阵,K、L的每一列为相空间在一组点在某个基函数上的取值,当相空间的点数足够多时,可以视矩阵的每一列为一个基函数的离散数据表示,K、L则可以看作是多个基函数的组合,分别表示演化前的基函数和演化后的基函数,而演化后的基函数又可以看作一个新的函数,视为Koopman算符作用在基函数上得到。于是K、L之间的关系可由Koopman算符表示
% UK = L
% Koopman算符的矩阵表示可由上式确定,因此我们可以通过上式求得Koopman算符的矩阵表示,进一步求得Koopman算符的本征值与本征函数。
% 1.2 常见的混沌系统 
% Logistic映射来源于生态学中的虫口模型,其动力学方程可描述为
% x_(n+1)=f(x_n )=rx_n (1-x_n ),x_n∈[0,1],n=1,2,3⋯
% 通过非线性动力学的知识可以得到,该系统在r取不同值时表现出不同的动力学行为。当0<r<1时,系统最终都会渐进的趋于0;当1<r<3时,系统会收敛到一个不动点,该不动点的值为x^*=(r-1)/r,此时系统的极限行为会趋于该不动点的值;当3<r<3.57时,系统的迭代会出现周期行为,随着r的增大,周期的长度也会相应的增加,例如2周期,4周期,8周期等;当3.57<r<4时,系统的迭代会在周期类型和混沌类型之间来回切换;直到r=4时,系统处于完全混沌的状态。在我们的Koopman分析中,我们取r=4的一个特例,通过Logistic映射的动力学方程演化出一系列的数据,作为Koopman分析的源数据,以此来分析Logistic映射的系统特征。
% Henon映射是由法国数学家Michel Hénon提出的,以此作为Lorenz模型的Poincare界面的简化模型。Henon映射是一个可以产生混沌现象的离散时间动态系统,其迭代方程可以描述为
% x_(n+1)=y_n+1-ax_n^2
% y_(n+1)=bx_n
% 在经典的Henon映射中,我们通常取a=1.4与b=0.3,此时系统表现出混沌现象。在我们的Koopman分析中,我们在取上述参数值的情况下,通过Henon映射的迭代方程演化出一系列的数据,作为Koopman分析的源数据,以此来分析Henon映射的系统特征。
% Lorenz系统是1963年由Edward Lorenz提出的描述空气流体运动的一个数学模型,通常用Lorenz方程来描述
% {█(x ̇=σ(y-x)        @y ̇=x(ρ-z)-y@z ̇=xy-βz          )┤
% 该方程描述了一个三维相空间的三个分量对时间的变化率,是一个连续时间动态系统。Lorenz系统具有非线性、非周期性和确定性的性质。参数σ,ρ,β取不同值时Lorenz系统表现出不同的动力学行为。如在ρ<1时,系统只有一个不动点,即原点,所有轨道的长期行为都趋于原点;当ρ=1时系统发生了叉式分叉,在ρ>1时出现了两个不动点,不动点的稳定性可由其他参数满足的条件确定;当我们将三个参数取一组特定的值ρ=28,σ=10,β=8/3时,Lorenz方程的解是混沌的,相空间会存在两个奇异吸引子。Lorenz系统的这种动力学特征用传统的非线性动力学知识是比较难分析的,我们将在取上述参数的条件下,通过Lorenz微分方程演化出相空间的一条轨道,作为Koopman分析的源数据,以此来分析Lorenz系统的特征。

\section{论文计划}
通过对上述一些常见的混沌系统的Koopman分析,我们可以构造出每个系统下的数据矩阵,进一步计算出Koopman算符的矩阵表示以及本征值、本征函数。每个本征函数都代表了该系统的一种动力学模式,我们可以得到每个系统的很多本征函数,且本征函数的数量与相空间的数据点的数量相同,然而很多本征函数都是高度抽象的,很多本征函数反映了相同的动力学模式,而我们的工作就是提取出每个系统不同的动力学模式。

在混沌系统中的Koopman分析可以为我们在以后实际应用中更复杂的非线性系统打下基础。通过已知的混沌系统的Koopman分析方式,我们可以观察到计算Koopman算符的本征函数时,不同的参数对本征函数的影响,比如基函数的形式、基函数的数量、基函数的分布、演化格点的分布等。在了解不同的参数对混沌系统的影响后,我们可以将Koopman分析方法应用到实际的复杂系统中。在实际的复杂系统中,我们可以采用与混沌系统同样的分析方法,即使系统非常复杂,我们也无需知道系统的工作内部机制以及动力学方程,只需要得到系统的状态演化过程,便能对复杂系统进行Koopman分析,并提取复杂系统的关键特征,从而分析复杂系统的长期行为,并对某些系统行为进行预测。

以Logistic映射为例,我们介绍Koopman分析的算法实现与具体步骤,不失一般性,我们取特定的参数r=4,此时Logistic映射表现为混沌状态,此时Logistic映射的动力学方程可以描述为

$$x_(n+1)=f(x_n )=4x_n (1-x_n ),x_n∈[0,1],n=1,2,3⋯$$

该系统为一维系统,相空间范围为[0,1],该系统存在两个不动点:0和3/4。我们首先根据该映射方程产生一组迭代数据,设我们产生的迭代数据的数量为n+1。则每个数据点可以表示为$x_i,i=1,2,⋯,n+1$,为了得到“演化前数据”和“演化后数据”,我们将这组迭代产生的数据构造为${x_1,x_2,⋯,x_n }与{x_2,x_3,⋯,x_(n+1) }$,并根据这两组数据构造出数据矩阵K与L。

在构造矩阵K与L时,我们需要选定一组基函数${g_j (x)},j=1,2,⋯m$,基函数的选取对我们计算本征函数的结果至关重要,于是我们将从多个角度讨论基函数的选取对本征函数的影响。基函数的选取至关重要,为了能较好的表示出Koopman算符的本征函数,需要一组较完备的基函数作为算符作用的可观测量,可以选取Gauss基函数、Fourier基函数、Legendre基函数。然而上述基函数只有在数量趋于无穷时才具有完备性,我们希望基函数的数量足够多,然而为了相空间的数据点能够较好的表示每个基函数,基函数的数量又不宜超过演化格点的数量,此外考虑到计算量的因素,我们只能将基函数的数量设为某个有限值。此外由于相空间粒子分布的不均匀性,我们还可以取不同分布的基函数。总之,基函数的选取有多重形式,而基函数的选取会影响到本征函数的计算,因此基函数的选取至关重要。

在选取合适的基函数与构造数据矩阵K与L后,我们即可以根据UK=L计算求得Koopman算符的矩阵表示以及本征值与本征函数,通常我们会得到n个本征值与本征函数,而在这些本征值中,我们比较关心的是本征值接近1的本征函数,因为本征值为1的本征函数反映了相空间中一条随时间演化不变的轨道,这种轨道即是系统的一个关键特征,而由于我们的数值计算误差,因此我们将选取接近1的本征值对应的本征函数。

其他混沌系统的分析方法与该一维的分析方法类似。

\section{论文进展情况}
在工作开始之前,我进行了相关准备工作,进行大量文献阅读和资料查询,对传统的动力学分析方法有了一定了解。为了简化高维系统,我们通常都会进行降维处理,比较常用的线性降维技术有主成份分析,独立成分分析等。主成份分析和独立成分分析实际上都是线性变换。但是现实中的系统都是非线性的,这样的线性的方法在某些时候是不适用的。进而有了本文的研究展开。

首先完成了对Koopman算符相关文献的调研,认识到对Koopman的大部分研究还处于黑盒子的状态,其次是了解到Koopman算符的普适性,可以应用在不同动力学系统上,无论是线性系统还是非线性系统,甚至混沌。然后我阅读了相关对某些动力学系统的Koopman分析与一些改进的Koopman算法。

在此基础上,我做了大量的仿真工作,分别讨论了在一维映射,二维映射,三维流下的Koopman算符的本征值与本征函数,较为全面的验证了Koopman算符的有效性。

\section{工作成果}
Logistic映射来源于生态学中的虫口模型,其动力学方程可描述为
$$x_(n+1)=f(x_n )=rx_n (1-x_n ),x_n∈[0,1],n=1,2,3⋯$$
通过非线性动力学的知识可以得到,该系统在r取不同值时表现出不同的动力学行为。当0<r<1时,系统最终都会渐进的趋于0;当1<r<3时,系统会收敛到一个不动点,该不动点的值为$x^*=(r-1)/$r,此时系统的极限行为会趋于该不动点的值;当3<r<3.57时,系统的迭代会出现周期行为,随着r的增大,周期的长度也会相应的增加,例如2周期,4周期,8周期等;当3.57<r<4时,系统的迭代会在周期类型和混沌类型之间来回切换;直到r=4时,系统处于完全混沌的状态。在我们的Koopman分析中,我们取r=4的一个特例,通过Logistic映射的动力学方程演化出一系列的数据,作为Koopman分析的源数据,以此来分析Logistic映射的系统特征。
  
其分叉图如上图所示,当γ>3.57时,系统呈现出混沌状态,不失一般性,我们取γ=4的混沌状态。
 
对其取高斯基函数(1000个演化格点,6个函数格点),计算得Koopman算符的最大本征值如下图,其中蓝色实线表示Koopman算符的最大实本征值对应的本征函数,方块点代表其本征函数的极小值的点,黄色点代表0以及0的原像点,绿色点代表0.75及其原像点,蓝色点表示周期2轨道,绿色点表示周期四轨道,粉色点表示周期8轨道。
从上图可以看出,本征函数的极值点与边界点和边界点的原像比较吻合,说明动力学系统的边界点代表着其某种动力学模式。
 
其实,动力学系统的边界点表示了其对相空间的划分,从符号动力学的角度来讲,每个相空间的点都可以用一个无限长的符号序列来描述,如上图所示,符号动力学将0.5左右分别标记为“0”和“1”,而每个划分区域的原像点又可以对该区域进行划分,使得相空间被划分为4个区域,如此继续区分下去,每个点都会对应一个唯一的符号序列,而该符号序列就是对该点粒子的运动做一定程度的预测。
 
计算Koopman算符的本征函数时,若基函数增大,会画得Koopman算符的本征函数如上图,总体来讲,Koopman算符本征函数的极小值与边界点扔吻合的较好,由此也可以见的Koopman算符的有效性。
 
Henon映射是由法国数学家Michel Hénon提出的,以此作为Lorenz模型的Poincare界面的简化模型。Henon映射是一个可以产生混沌现象的离散时间动态系统,其迭代方程可以描述为
$$x_(n+1)=y_n+1-ax_n^2$$
$$y_(n+1)=bx_n$$
在经典的Henon映射中,我们通常取a=1.4与b=0.3,此时系统表现出混沌现象。在我们的Koopman分析中,我们在取上述参数值的情况下,通过Henon映射的迭代方程演化出一系列的数据,作为Koopman分析的源数据,以此来分析Henon映射的系统特征。
 
Henon映射的吸引子图像与不动点位置如上图所示,对Henon映射的Koopman分析使用做二维高斯基函数,计算得Koopman算符的本征函数如下图。。
 
其中彩色背景表示Koopman算符本征值的绝对值大小,中间的粉色线表示Henon映射的吸引子,吸引子上的点为Henon映射的周期点,周期点上会存在稳定流形和不稳定流形的方向,由此可见,Koopman算符能较好的反映二维映射的周期信息。
 
根据周期点上的稳定流行与不稳定流形的演化,我们甚至可以得出不稳定流行的完整图像,我们会发现,不稳定流行的图像会和本征函数体现出高度的吻合性。
Lorenz系统是1963年由Edward Lorenz提出的描述空气流体运动的一个数学模型,通常用Lorenz方程来描述
$$x=y$$
该方程描述了一个三维相空间的三个分量对时间的变化率,是一个连续时间动态系统。Lorenz系统具有非线性、非周期性和确定性的性质。参数σ,ρ,β取不同值时Lorenz系统表现出不同的动力学行为。如在ρ<1时,系统只有一个不动点,即原点,所有轨道的长期行为都趋于原点;当ρ=1时系统发生了叉式分叉,在ρ>1时出现了两个不动点,不动点的稳定性可由其他参数满足的条件确定;当我们将三个参数取一组特定的值ρ=28,σ=10,β=8/3时,Lorenz方程的解是混沌的,相空间会存在两个奇异吸引子。Lorenz系统的这种动力学特征用传统的非线性动力学知识是比较难分析的,我们将在取上述参数的条件下,通过Lorenz微分方程演化出相空间的一条轨道,作为Koopman分析的源数据,以此来分析Lorenz系统的特征。
 
上图为Lorenz系统的相图,我们对其选择三维高斯基函数,计算得Koopman算符的本征函数如下图,其中每个点都为Lorenz系统相图的一点,颜色代表Koopman算符本征函数值的大小。可以在图像中观察到一定的周期结构,这也跟本征函数幅角的周期变化有关。
 
 
若使用本征函数的正负值来区别Lorenz系统的相空间,我们可以将Lorenz系统分成两个部分,理论上这两部分即为Lorenz系统的不变集。

\section{计划及进度安排}
2018年12月至2019年5月,查找相关文献资料,对Koopman算符的研究成果进行分析,熟悉别人对这方面做出的成果。
2019年5月至2019年7月,阅读动力学系统的性质等相关书籍与文献,分析Koopman算符是如何描述动力学系统的。
2019年7月至2019年9月,阅读混沌系统与其动力学性质等相关文献,并分析混沌系统中的Koopman算符。
2019年9月至2019年11月,阅读DMD、Koopman算符相关文献,基于DMD对混沌系统进行Koopman分析,开始撰写论文。
2019年11月至2019年12月,整理并总结不同混沌系统中Koopman算符的差异,进行对比分析,完善论文。
2019年12月至2020年3月,仿真分析进一步整理完善,完成论文撰写。
2020年3月至6月,进行论文修改,毕业答辩。

\section{问题及整改方案}
1.如何精确的用一些随机梯度下降算法求得精简的本征值。以及如何跳出局部最优解。
2.随着基函数的增多,本征函数的划分也变得更精确,如何通过粗粒度的划分来实现细粒度的划分。
3.如果原方程加了噪声,会对Koopman算符本征函数有何影响。
4.Koopman算符的本征函数是否具有鲁棒性。
5.如何确定边界点的层次。
6.能否用通用的方法画出动力学系统的分界线。
7.本征函数的最大值最小值分别表示什么含义。
8.基函数的变化是如何影响本征函数的。
以上都是在课题中遇到的问题,今后会继续加强对这些问题的思考与完善,并将这些思考过程择重点记录到论文中。




%% 本章参考文献
% \ifx\usechapbib\empty
% \nocite{BSTcontrol}
% \setcounter{NAT@ctr}{0}
% \bibliographystyle{buptgraduatethesis}
% \bibliography{references}
% \fi
