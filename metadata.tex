%%
%% This is file `example/metadata.tex',
%% generated with the docstrip utility.
%%
%% The original source files were:
%%
%% install/buptgraduatethesis.dtx  (with options: `metadata')
%% 
%% This file is a part of the example of BUPTGraduateThesis.
%% 

%% 涉密论文保密年限
\classdur{三年}

%% 学号
\studentid{2017110978}

%% 论文题目
\ctitle{混沌系统的Koopman分析与应用}
\etitle{Koopman Analysis and Application of Chaotic Map}

%% 申请学位
\cdegree{理学硕士}

%% 院系名称
\cdepartment{理学院}
\edepartment{School of Science}

%% 专业名称
\cmajor{系统科学}
\emajor{System Science}

%% 你的姓名
\cauthor{张聪}
\eauthor{Zhang Cong}

%% 博士后研究工作报告-分类号
\classnumber{O441.3}

%% 博士后研究工作报告-UDC
\udc{621.396.9}

%% 博士后研究工作报告-学校编号
\schoolserial{147227}

%% 博士后研究工作起始时间
\startdate{2014年10月29日}

%% 博士后研究工作期满时间
\finishdate{2016年4月2日}

%% 你导师的姓名
\csupervisor{兰岳恒}
\esupervisor{Lan Yueheng}

%% 日期自动生成,也可以取消注释下面一行,自行指定日期

%% 中文摘要
\cabstract{%
现实中的大多数系统往往由于原理复杂而难以用比较准确的动力学方程来近似描述,只能通常用大量的实验观测得到系统的特性数据。我们希望能找到一个普遍的数据分析方法,通过系统的特性数据,提取出动力学系统的动力学模式。Koopman算符提供了一个有效的数学工具,它作用在某个函数上,描述了函数的演化,若将系统的特性数据演化视为函数的演化,我们即可以用Koopman算符分析系统的演化特征,并进一步提取出关键特征,并可以对系统的长期行为作一定程度的预测。我们在一些常见的混沌系统上(例如Logistic映射,Henon映射,Lorenz系统)应用Koopman算符进行谱分解,有效地提取出混沌系统的特征,并对Koopman算符的本征值与本征函数作了一些物理解释。由于Koopman算符的普适性,我们可以将Koopman分析应用到一般的复杂系统中。
}

%% 中文关键词,关键词之间用 \kwsep 分割
\ckeywords{Koopman算符 \kwsep 动力学模式 \kwsep 谱分解 \kwsep 混沌映射}

%% 英文摘要
\eabstract{%
Most systems in reality are often difficult to approximate with accurate dynamical equations due to their complexity, and experimental observations can be used solely to obtain system characteristic data. We hope to find a universal data analysis method to extract the dynamical model underlying a nonlinear system through the characteristic data. The Koopman operator provides an effective mathematical tool, which acts on certain functions and describes their evolution. Based on the time series of the system's evolution, we can use the Koopman operator to analyze the temporal characteristics of the system, and extract key dynamical factors, and predict the long-term behavior of the system to certain extent. We apply the Koopman operator technique to spectral decomposition of several typical chaotic systems (e.g. Logistic map, Henon map and Lorenz system), effectively extract their key features, and explain the eigenvalues and eigenfunctions of the Koopman operator. Because of the universality of the analysis, we may apply it to general complex systems.
}

%% 英文关键词,也用 \kwsep 分割
\ekeywords{%
  Koopman Operater \kwsep Dynamic model \kwsep Spectral Decomposition \kwsep Chaotic Systems}
